\documentclass[12pt,a4paper]{article}
\usepackage[utf8]{inputenc}
\usepackage[spanish]{babel}
\usepackage{amsmath, amssymb, amsfonts}
\usepackage{geometry}
\usepackage{graphicx}
\usepackage{hyperref}
\usepackage{listings}
\usepackage{xcolor}
\usepackage{float}
\usepackage{fancyhdr}
\usepackage{titlesec}
\usepackage{enumitem}
\usepackage{array}
\usepackage{booktabs}

\geometry{left=2.5cm,right=2.5cm,top=2.5cm,bottom=2.5cm}

% Configuracion de colores para codigo
\definecolor{codegreen}{rgb}{0,0.6,0}
\definecolor{codegray}{rgb}{0.5,0.5,0.5}
\definecolor{codepurple}{rgb}{0.58,0,0.82}
\definecolor{backcolour}{rgb}{0.95,0.95,0.92}

% Configuracion de listings
\lstdefinestyle{mystyle}{
    backgroundcolor=\color{backcolour},   
    commentstyle=\color{codegreen},
    keywordstyle=\color{magenta},
    numberstyle=\tiny\color{codegray},
    stringstyle=\color{codepurple},
    basicstyle=\ttfamily\footnotesize,
    breakatwhitespace=false,         
    breaklines=true,                 
    captionpos=b,                    
    keepspaces=true,                 
    numbers=left,                    
    numbersep=5pt,                  
    showspaces=false,                
    showstringspaces=false,
    showtabs=false,                  
    tabsize=2
}
\lstset{style=mystyle}

% Configuracion de headers
\pagestyle{fancy}
\fancyhf{}
\rhead{Modelado de Trayectorias Elipticas para Drones}
\lhead{Universidad Catolica de Temuco}
\cfoot{\thepage}

% PORTADA
\begin{document}
\begin{titlepage}
    \centering
    \vspace*{2cm}
    
    {\LARGE\textbf{UNIVERSIDAD CATOLICA DE TEMUCO}}\\[0.5cm]
    {\large Ingenieria Civil en Informatica}\\[2cm]
    {\large MAT1186 Introduccion al Calculo}\\[2cm]
    
    \rule{\linewidth}{0.2mm} \\[0.4cm]
    {\huge\bfseries MODELADO DE TRAYECTORIAS ELIPTICAS PARA DRONES}\\[0.2cm]
    \rule{\linewidth}{0.2mm} \\[1.5cm]

    {\large Profesor Gustavo Sandoval Cofré}\\[0.5cm]
    {\large Esban Vejar Chavez, Joaquin Cantero Olivera, Oscar Zapata Benavides y Fabian Garcia Valdebenito}\\[0.5cm]

    {\large 06 de Junio de 2025}\\
    
    \vfill
\end{titlepage}

% INDICE
\tableofcontents
\newpage

% INTRODUCCION
\section{Introduccion}

En la era actual de la automatizacion y la inteligencia artificial, los sistemas de drones autonomos han emergido como una tecnologia revolucionaria con aplicaciones que abarcan desde la logistica y el transporte hasta la agricultura de precision y la vigilancia de seguridad. Sin embargo, el crecimiento exponencial en el uso de estos vehiculos aereos no tripulados (UAV) ha traido consigo desafios significativos en terminos de navegacion segura y prevencion de colisiones en espacios aereos compartidos.\\

El modelado matematico de trayectorias representa un componente fundamental en el desarrollo de sistemas de navegacion autonoma confiables. Entre las diversas formas geometricas utilizadas para describir rutas de vuelo, las trayectorias elipticas ofrecen ventajas particulares debido a su capacidad para representar patrones de movimiento naturales y eficientes desde el punto de vista energetico, caracteristicas especialmente relevantes en aplicaciones donde la autonomia de bateria es critica.

\subsection{Problematica Abordada}

El problema central que aborda este proyecto es la necesidad de un sistema robusto y eficiente para:

\begin{enumerate}
    \item Generar trayectorias elipticas personalizadas basadas en identificadores unicos (RUT chileno).
    \item Detectar potenciales conflictos entre multiples trayectorias en tiempo real.
    \item Visualizar y analizar escenarios de riesgo para la toma de decisiones operacionales.
    \item Proporcionar una base matematica solida para futuros desarrollos en sistemas de gestion de trafico aereo no tripulado.
\end{enumerate}

\subsection{Alcance y Objetivos}

Este informe presenta el desarrollo e implementacion de un sistema computacional que integra modelado matematico avanzado con tecnicas de visualizacion y analisis de datos, especificamente disenado para:\\

\textbf{Objetivo General:}
Desarrollar un sistema integral de modelado y analisis de trayectorias elipticas para drones autonomos que garantice operaciones seguras y eficientes en entornos de multiples vehiculos.\\

\textbf{Objetivos Especificos:}
\begin{itemize}
    \item Implementar algoritmos de generacion de trayectorias elipticas basados en parametros derivados de identificadores unicos.
    \item Desarrollar algoritmos de deteccion de colisiones que identifiquen intersecciones y zonas de riesgo entre trayectorias.
    \item Crear interfaces de usuario intuitivas para la visualizacion y analisis de escenarios complejos.
    \item Validar la efectividad del sistema mediante pruebas con multiples configuraciones de trayectorias.
    \item Establecer las bases para futuras aplicaciones en gestion de trafico aereo no tripulado.
\end{itemize}

\section{Desarrollo}

\subsection{Marco Teorico}

\subsubsection{Fundamentos Matematicos de las Trayectorias Elipticas}

Una elipse en el plano cartesiano puede definirse como el lugar geometrico de todos los puntos cuya suma de distancias a dos puntos fijos (focos) es constante. Matematicamente, una elipse con centro en $(h,k)$ y orientacion $\theta$ se describe mediante la ecuacion parametrica:

\begin{align}
x(t) &= h + a\cos(t)\cos(\theta) - b\sin(t)\sin(\theta) \\
y(t) &= k + a\cos(t)\sin(\theta) + b\sin(t)\cos(\theta)
\end{align}

donde:
\begin{itemize}
    \item $(h,k)$ representa las coordenadas del centro de la elipse
    \item $a$ es la longitud del semieje mayor
    \item $b$ es la longitud del semieje menor
    \item $\theta$ es el angulo de rotacion respecto al eje x
    \item $t \in [0, 2\pi]$ es el parametro que describe la posicion a lo largo de la trayectoria
\end{itemize}

La ecuacion canonica de una elipse se expresa como:

\[\frac{(x-h)^2}{a^2} + \frac{(y-k)^2}{b^2} = 1\]

Esta representacion matematica permite calcular de manera eficiente:
\begin{itemize}
    \item Posicion instantanea del drone en cualquier momento $t$
    \item Velocidad y aceleracion mediante derivacion de las ecuaciones parametricas
    \item Distancias y tiempos de transito entre puntos especificos
    \item Verificacion de pertenencia de puntos a la trayectoria
\end{itemize}

\subsubsection{Generacion de Parametros desde RUT}

El sistema implementa un algoritmo innovador que transforma el Rol Unico Tributario (RUT) chileno en parametros elipticos unicos y reproducibles. Este enfoque garantiza que cada usuario del sistema genere trayectorias consistentes y personalizadas.\\

El algoritmo de transformacion sigue estos pasos:

\begin{enumerate}
    \item \textbf{Validacion del formato}: Verificacion de la estructura RUT (XXXXXXXX-Y)
    \item \textbf{Extraccion de digitos}: Separacion de digitos significativos del digito verificador
    \item \textbf{Calculo del grupo}: Determinacion basada en el digito verificador (par/impar)
    \item \textbf{Asignacion de parametros}: Mapeo sistematico de digitos a coordenadas y dimensiones
\end{enumerate}

La logica de asignacion se implementa como:

\begin{lstlisting}[language=Python, caption=Algoritmo de generacion de parametros elipticos]
def desde_rut(cls, rut: str):
    # Procesamiento y validacion del RUT
    partes = rut.strip().split('-')
    rut_sin_dv = partes[0].replace('.', '')
    dv = partes[1].upper()
    
    # Extraccion de digitos significativos
    digits = [int(d) for d in rut_sin_dv if d.isdigit()]
    dv_int = int(dv) if dv != 'K' else 10
    
    # Asignacion de coordenadas del centro
    h, k = digits[0], digits[1]
    
    # Calculo de semiejes segun paridad del DV
    if dv_int % 2 == 1:  # DV impar
        a = digits[2] + digits[3]
        b = digits[4] + digits[5]
        theta = 0 if digits[5] % 2 == 0 else np.pi / 2
    else:  # DV par
        a = digits[5] + digits[6]
        b = digits[4] + digits[2]
        theta = 0 if digits[3] % 2 == 0 else np.pi / 2
    
    return cls(h, k, a, b, theta, rut)
\end{lstlisting}

\subsubsection{Algoritmos de Deteccion de Colisiones}

La deteccion de colisiones constituye el nucleo del sistema de seguridad. Se implementaron tres niveles de analisis:\\

\textbf{1. Deteccion Basica de Interseccion}\\

El algoritmo fundamental verifica si puntos de una trayectoria eliptica se encuentran dentro del area definida por otra elipse:

\begin{lstlisting}[language=Python, caption=Algoritmo de deteccion basica]
def hay_colision_trayectorias(elipse1, elipse2, n=200):
    # Verificar puntos de elipse1 dentro de elipse2
    x1, y1 = elipse1.puntos(n)
    for xi, yi in zip(x1, y1):
        if elipse2.contiene_punto(xi, yi):
            return True
    
    # Verificar puntos de elipse2 dentro de elipse1
    x2, y2 = elipse2.puntos(n)
    for xi, yi in zip(x2, y2):
        if elipse1.contiene_punto(xi, yi):
            return True
    
    return False
\end{lstlisting}

\textbf{2. Identificacion de Rutas de Cruce}\\

Este algoritmo avanzado mapea todas las regiones donde las trayectorias se solapan:

\begin{lstlisting}[language=Python, caption=Algoritmo de rutas de cruce]
def ruta_cruce(e1, e2, n=500, tol=1e-5):
    puntos = []
    
    # Recopilar puntos de interseccion
    x1, y1 = e1.puntos(n)
    for xi, yi in zip(x1, y1):
        if e2.contiene_punto(xi, yi):
            puntos.append((xi, yi))
    
    x2, y2 = e2.puntos(n)
    for xi, yi in zip(x2, y2):
        if e1.contiene_punto(xi, yi):
            puntos.append((xi, yi))
    
    # Filtrar puntos duplicados
    filtrados = []
    for px, py in puntos:
        if not any(np.hypot(px - fx, py - fy) < tol 
                  for fx, fy in filtrados):
            filtrados.append((px, py))
    
    return filtrados
\end{lstlisting}

\textbf{3. Localizacion de Puntos Criticos}\\

Identifica ubicaciones especificas donde las trayectorias se intersectan:

\begin{lstlisting}[language=Python, caption=Algoritmo de puntos de interseccion]
def puntos_interseccion_aproximados(e1, e2, n=500):
    puntos_cruce = []
    
    # Analizar transiciones de contencion
    x1, y1 = e1.puntos(n)
    dentro_anterior = e2.contiene_punto(x1[0], y1[0])
    
    for i in range(1, n):
        dentro_actual = e2.contiene_punto(x1[i], y1[i])
        if dentro_actual != dentro_anterior:
            puntos_cruce.append((x1[i], y1[i]))
        dentro_anterior = dentro_actual
    
    return puntos_cruce
\end{lstlisting}

\subsection{Arquitectura del Sistema}

\subsubsection{Diseno Modular}

El sistema se estructura siguiendo principios de ingenieria de software modular, organizandose en las siguientes capas:

\textbf{Capa de Modelos}
\begin{itemize}
    \item \texttt{TrayectoriaEliptica}: Clase principal que encapsula la logica matematica de las trayectorias
    \item Metodos para calculo de posicion, velocidad y verificacion de pertenencia
    \item Interfaz para generacion desde RUT y representacion grafica
\end{itemize}

\textbf{Capa de Servicios}
\begin{itemize}
    \item \texttt{ColisionadorTrayectorias}: Servicio especializado en analisis de conflictos
    \item Algoritmos optimizados para deteccion en tiempo real
    \item Gestion de multiples trayectorias simultaneas
\end{itemize}

\textbf{Capa de Utilidades}
\begin{itemize}
    \item \texttt{graficos\_utils}: Funciones de visualizacion y renderizado
    \item \texttt{colision\_utils}: Algoritmos de bajo nivel para calculos geometricos
    \item \texttt{trayectoria\_utils}: Utilidades matematicas auxiliares
\end{itemize}

\textbf{Capa de Presentacion}
\begin{itemize}
    \item Interfaz grafica basada en PyQt6 para usuarios avanzados (sistema 2D)
    \item Interfaz Tkinter con canvas 3D para visualizacion espacial avanzada
    \item Interfaz de linea de comandos para operaciones automatizadas
    \item Sistema de visualizacion interactiva con Matplotlib y capacidades 3D
\end{itemize}

\subsubsection{Arquitectura del Sistema 3D}

La implementacion del sistema de visualizacion 3D introduce una arquitectura especializada que complementa el sistema 2D existente:

\textbf{Componentes del Sistema 3D}
\begin{itemize}
    \item \textbf{Motor de Transformacion 3D}: Algoritmos para conversion de parametros elipticos a coordenadas tridimensionales
    \item \textbf{Gestor de Renderizado}: Integracion entre Tkinter y Matplotlib para visualizacion 3D
    \item \textbf{Procesador de Intersecciones 3D}: Algoritmos adaptados para deteccion de conflictos en el espacio
    \item \textbf{Controlador de Interfaz}: Gestion de eventos y actualizaciones en tiempo real
\end{itemize}

\textbf{Flujo de Datos en el Sistema 3D}

\begin{lstlisting}[language=Python, caption=Flujo principal del sistema 3D]
def sistema_3d_workflow():
    # 1. Validacion de entrada
    rut = validar_formato_rut(entrada_usuario)
    
    # 2. Transformacion a parametros 3D
    x, y, z, params = compute_ellipse(rut)
    
    # 3. Deteccion de conflictos espaciales
    if len(trayectorias) >= 2:
        intersecciones = detectar_intersecciones_3d(
            trayectorias[-2], trayectorias[-1])
    
    # 4. Renderizado tridimensional
    canvas_3d = crear_canvas_matplotlib()
    renderizar_trayectorias_3d(trayectorias, canvas_3d)
    
    # 5. Actualizacion de interfaz
    actualizar_panel_informacion(params, intersecciones)
\end{lstlisting}

\textbf{Optimizaciones de Rendimiento}

El sistema 3D implementa varias optimizaciones para garantizar rendimiento en tiempo real:

\begin{itemize}
    \item \textbf{Sampling adaptativo}: Ajuste dinamico del numero de puntos segun la complejidad de la escena
    \item \textbf{Caching de calculos}: Almacenamiento temporal de transformaciones costosas
    \item \textbf{Renderizado incremental}: Actualizacion solo de elementos modificados
    \item \textbf{Gestion de memoria}: Liberacion automatica de recursos graficos no utilizados
\end{itemize}

\subsubsection{Implementacion de Interfaces}

\textbf{Interfaz Grafica de Usuario (GUI)}

La GUI implementada en PyQt6 proporciona una experiencia de usuario intuitiva y profesional:

\begin{itemize}
    \item \textbf{Panel de entrada}: Campo para ingreso de RUT con validacion en tiempo real
    \item \textbf{Lista de trayectorias}: Visualizacion tabular de todas las trayectorias activas
    \item \textbf{Area de visualizacion}: Canvas de Matplotlib integrado para graficos 2D
    \item \textbf{Controles de analisis}: Botones para deteccion global y analisis especifico
    \item \textbf{Ventanas modales}: Dialogos especializados para analisis detallado
\end{itemize}

La arquitectura de la GUI sigue el patron Model-View-Controller (MVC):

\begin{lstlisting}[language=Python, caption=Estructura principal de la GUI]
class MainWindow(QWidget):
    def __init__(self):
        super().__init__()
        self.trayectorias = []  # Modelo de datos
        self.setup_ui()         # Configuracion de vista
        self.connect_events()   # Configuracion de controladores
    
    def agregar_trayectoria(self):
        # Validacion y creacion de nueva trayectoria
        
    def graficar_trayectorias(self):
        # Renderizado de visualizacion 2D
        
    def buscar_colisiones(self):
        # Ejecucion de analisis de conflictos
\end{lstlisting}

\textbf{Interfaz de Linea de Comandos (CLI)}

La CLI ofrece funcionalidad completa para usuarios tecnicos y sistemas automatizados:

\begin{itemize}
    \item Menu interactivo con navegacion numerica
    \item Comandos para todas las operaciones principales
    \item Salida formateada para analisis y reporting
    \item Compatibilidad con scripts y automatizacion
\end{itemize}

\subsection{Algoritmos de Visualizacion}

\subsubsection{Renderizado de Trayectorias en 2D}

El sistema de visualizacion utiliza Matplotlib para generar representaciones graficas precisas y esteticamente atractivas:

\begin{lstlisting}[language=Python, caption=Funcion de graficado de elipses 2D]
def graficar_elipses(trayectorias, ax, colores=None, 
                    with_labels=True, linewidth=2):
    if colores is None:
        colores = ["blue", "green", "orange", "purple", 
                  "cyan", "brown", "magenta", "gold", "pink"]
    
    for i, tr in enumerate(trayectorias, 1):
        color = colores[(i - 1) % len(colores)]
        x, y = tr.puntos()
        label = f"RUT: {tr.rut}" if with_labels else None
        ax.plot(x, y, color=color, linewidth=linewidth, label=label)
    
    if with_labels:
        ax.legend()
    ax.set_aspect('equal')
\end{lstlisting}

\subsubsection{Desarrollo de Visualizacion 3D}

Para complementar el analisis 2D tradicional, se desarrollo un sistema de visualizacion tridimensional que permite una comprension mas profunda de las trayectorias elipticas en el espacio. Esta implementacion utiliza Tkinter como framework base y Matplotlib para el renderizado 3D.\\

\textbf{Arquitectura del Sistema 3D}\\

El sistema 3D se estructura en cuatro componentes principales:

\begin{enumerate}
    \item \textbf{Panel de entrada}: Validacion y procesamiento de RUT con expresiones regulares
    \item \textbf{Panel de gestion}: Visualizacion de RUTs almacenados y estado del sistema
    \item \textbf{Area de renderizado}: Canvas 3D interactivo con controles de navegacion
    \item \textbf{Panel de informacion}: Datos calculados y resultados de analisis
\end{enumerate}

\textbf{Algoritmo de Transformacion 3D}\\

La extension tridimensional introduce una componente Z calculada dinamicamente:

\begin{lstlisting}[language=Python, caption=Generacion de elipses 3D]
def compute_ellipse(rut):
    numeros, verificador = rut
    digitos = list(map(int, numeros))
    es_impar = verificador.isdigit() and int(verificador) % 2 == 1

    if es_impar:
        h, k = digitos[0], digitos[1]
        a = digitos[2] + digitos[3]
        b = digitos[4] + digitos[5]
        orientacion_horizontal = (digitos[7] % 2 == 0)
    else:
        h, k = digitos[0], digitos[1]
        a = digitos[5] + digitos[6]
        b = digitos[7] + digitos[2]
        orientacion_horizontal = (digitos[3] % 2 == 0)
    
    theta = np.linspace(0, 2 * np.pi, 100)
    x = a * np.cos(theta)
    y = b * np.sin(theta)
    z = np.sin(theta) * (a + b) / 2  # Componente altura
    
    # Aplicar orientacion
    if not orientacion_horizontal:
        x, y = y, x

    # Trasladar al centro
    x += h
    y += k
    
    return x, y, z, {"h": h, "k": k, "a": a, "b": b, 
                    "rut": f"{numeros}-{verificador}"}
\end{lstlisting}

\textbf{Deteccion de Intersecciones 3D}\\

El algoritmo de deteccion se adapta para trabajar con proyecciones 2D de las trayectorias 3D:

\begin{lstlisting}[language=Python, caption=Algoritmo de intersecciones 3D]
def detectar_intersecciones(x1, y1, x2, y2, tol=0.5):
    intersecciones = []
    for i in range(len(x1)):
        for j in range(len(x2)):
            d = sqrt((x1[i] - x2[j])**2 + (y1[i] - y2[j])**2)
            if d < tol:
                xi = (x1[i] + x2[j]) / 2
                yi = (y1[i] + y2[j]) / 2
                intersecciones.append((round(xi,2), round(yi,2)))
    
    # Eliminar duplicados
    unique_inters = []
    for pt in intersecciones:
        if pt not in unique_inters:
            unique_inters.append(pt)
    return unique_inters
\end{lstlisting}

\textbf{Renderizado y Configuracion del Canvas 3D}\\

La implementacion del renderizado 3D utiliza las capacidades avanzadas de Matplotlib:

\begin{lstlisting}[language=Python, caption=Configuracion del renderizado 3D]
def graficar_elipses():
    if len(rut_lista) < 2:
        messagebox.showerror("Error", 
                           "Se deben ingresar al menos 2 RUTs antes de graficar")
        return
    
    # Calcular las dos ultimas elipses
    x1, y1, z1, params1 = compute_ellipse(rut_lista[-2])
    x2, y2, z2, params2 = compute_ellipse(rut_lista[-1])

    # Configurar figura 3D
    fig = plt.figure(figsize=(8, 6))
    ax = fig.add_subplot(111, projection='3d')
    
    # Renderizar trayectorias
    ax.plot(x1, y1, z1, label=f"Elipse 1 (RUT: {params1['rut']})", 
            color="#E74C3C")
    ax.plot(x2, y2, z2, label=f"Elipse 2 (RUT: {params2['rut']})", 
            color="#3498DB")
    
    # Marcar centros
    ax.scatter([params1["h"]], [params1["k"]], [0], 
              color='red', marker='o', s=50, label="Centro 1")
    ax.scatter([params2["h"]], [params2["k"]], [0], 
              color='blue', marker='o', s=50, label="Centro 2")
    
    # Detectar y marcar intersecciones
    inters = detectar_intersecciones(x1, y1, x2, y2, tol=0.5)
    if inters:
        if len(inters) > 2:
            inters = inters[:2]
        inters_arr = np.array(inters)
        ax.scatter(inters_arr[:,0], inters_arr[:,1], 
                  [0]*len(inters_arr), color="red", marker="o", 
                  s=80, label="Intersecciones")
    
    # Configuracion de ejes y etiquetas
    ax.set_xlabel("X")
    ax.set_ylabel("Y")
    ax.set_zlabel("Altura (Z)")
    ax.legend(loc="upper right")
    ax.set_title("Elipses 3D generadas")
    ax.grid()
    
    # Integrar con canvas de Tkinter
    canvas = FigureCanvasTkAgg(fig, master=frame_grafico)
    canvas.get_tk_widget().pack()
    canvas.draw()
\end{lstlisting}

\textbf{Interfaz de Usuario 3D}\\

La interfaz grafica 3D implementa un diseno de cuatro paneles distribuidos estrategicamente:

\begin{lstlisting}[language=Python, caption=Configuracion de la interfaz 3D]
# Configuracion de ventana principal
ventana = tk.Tk()
ventana.title("Ingreso de RUT y Grafica de Elipses en 3D")
ventana.geometry("1200x800")

# Panel de entrada (superior izquierda)
frame_input = tk.Frame(ventana, relief="ridge", bd=5)
frame_input.place(x=20, y=20, width=750, height=160)

# Panel de RUTs (superior derecha)
frame_ruts = tk.Frame(ventana, relief="ridge", bd=5)
frame_ruts.place(x=800, y=20, width=350, height=160)

# Panel grafico (inferior izquierda)
frame_grafico = tk.Frame(ventana, relief="ridge", bd=5)
frame_grafico.place(x=20, y=200, width=750, height=550)

# Panel de datos (inferior derecha)
frame_datos = tk.Frame(ventana, relief="ridge", bd=5)
frame_datos.place(x=800, y=200, width=350, height=550)
\end{lstlisting}

\subsubsection{Visualizacion de Conflictos}

La representacion visual de colisiones utiliza un esquema de colores intuitivo tanto en 2D como en 3D:

\begin{itemize}
    \item \textbf{Azul/Verde}: Trayectorias normales sin conflictos
    \item \textbf{Rojo}: Rutas de cruce identificadas y puntos de interseccion
    \item \textbf{Negro}: Puntos criticos de interseccion (version 2D)
    \item \textbf{Magenta/Naranja}: Centros de trayectorias
    \item \textbf{Degradados}: Variaciones de altura en visualizacion 3D
\end{itemize}

\subsection{Validacion y Pruebas}

\subsubsection{Validacion del Sistema 3D}

El desarrollo del sistema de visualizacion 3D introdujo nuevos desafios de validacion que requirieron la implementacion de pruebas especializadas:

\begin{enumerate}
    \item \textbf{Pruebas de consistencia dimensional}:
    \begin{itemize}
        \item Verificacion de que las proyecciones 2D mantienen coherencia con las trayectorias originales
        \item Validacion de que la componente Z se calcula correctamente segun los parametros elipticos
        \item Pruebas de continuidad en las transiciones entre puntos 3D
    \end{itemize}
    
    \item \textbf{Pruebas de renderizado}:
    \begin{itemize}
        \item Verificacion de la correcta integracion entre Tkinter y Matplotlib
        \item Pruebas de rendimiento con multiples trayectorias simultaneas
        \item Validacion de la interactividad del canvas 3D (zoom, rotacion, pan)
    \end{itemize}
    
    \item \textbf{Pruebas de deteccion 3D}:
    \begin{itemize}
        \item Comparacion entre detecciones 2D y proyecciones 3D
        \item Validacion de tolerancias en el espacio tridimensional
        \item Pruebas de precision en el calculo de intersecciones
    \end{itemize}
\end{enumerate}

\subsubsection{Casos de Prueba del Sistema General}

Se desarrollo una suite completa de casos de prueba para validar la funcionalidad del sistema:

\begin{enumerate}
    \item \textbf{Pruebas de validacion de RUT}:
    \begin{itemize}
        \item RUTs validos con diferentes formatos
        \item RUTs invalidos con caracteres incorrectos
        \item Casos limite con longitudes minimas y maximas
    \end{itemize}
    
    \item \textbf{Pruebas de generacion de trayectorias}:
    \begin{itemize}
        \item Consistencia en la generacion (mismo RUT = misma trayectoria)
        \item Diversidad (RUTs diferentes = trayectorias diferentes)
        \item Validez matematica de los parametros generados
    \end{itemize}
    
    \item \textbf{Pruebas de deteccion de colisiones}:
    \begin{itemize}
        \item Trayectorias claramente separadas (sin colision)
        \item Trayectorias con interseccion parcial
        \item Trayectorias concentricas (una dentro de otra)
        \item Casos limite con tangencias
    \end{itemize}
\end{enumerate}

\subsection{Aplicaciones en Ingenieria}

\subsubsection{Ventajas de la Visualizacion 3D en Sistemas de Drones}

La implementacion de visualizacion tridimensional aporta ventajas significativas para aplicaciones reales de drones:

\begin{itemize}
    \item \textbf{Comprension espacial mejorada}: La visualizacion 3D permite a los operadores comprender mejor las relaciones espaciales entre trayectorias, especialmente en operaciones que involucran diferentes altitudes de vuelo.
    
    \item \textbf{Planificacion de vuelo optimizada}: La componente Z calculada dinamicamente proporciona informacion valiosa para la planificacion de rutas que consideran variaciones de altura y obstaculos topograficos.
    
    \item \textbf{Deteccion de conflictos espaciales}: El sistema 3D permite identificar situaciones donde las trayectorias se cruzan en diferentes planos, proporcionando una evaluacion mas completa de los riesgos potenciales.
    
    \item \textbf{Interfaz intuitiva}: La navegacion interactiva (zoom, rotacion, pan) facilita el analisis detallado de escenarios complejos desde multiples perspectivas.
\end{itemize}

\subsubsection{Gestion de Trafico Aereo No Tripulado}

El sistema desarrollado tiene aplicaciones directas en la gestion de trafico aereo para drones, especialmente con las capacidades 3D:

\begin{itemize}
    \item \textbf{Control de trafico}: Monitoreo en tiempo real de multiples vehiculos con separacion vertical
    \item \textbf{Planificacion de rutas}: Optimizacion de trayectorias considerando perfiles de altitud
    \item \textbf{Respuesta a emergencias}: Deteccion temprana de conflictos tridimensionales
    \item \textbf{Coordinacion de flotas}: Sincronizacion de operaciones en diferentes niveles de vuelo
\end{itemize}

\subsubsection{Agricultura de Precision}

En aplicaciones agricolas, el sistema puede coordinar multiples drones considerando topografia y variaciones de altura:

\begin{itemize}
    \item Fumigacion sincronizada adaptada a relieves topograficos
    \item Monitoreo simultaneo con sensores considerando altitudes optimas
    \item Optimizacion de cobertura evitando solapamientos en terrenos irregulares
    \item Coordinacion de drones de diferentes capacidades con perfiles de vuelo especializados
\end{itemize}

\subsubsection{Logistica y Transporte}

Para operaciones logisticas urbanas, la visualizacion 3D aporta:

\begin{itemize}
    \item Rutas de entrega optimizadas considerando edificaciones y obstaculos urbanos
    \item Coordinacion de multiples puntos de distribucion en diferentes altitudes
    \item Gestion de corredores aereos urbanos con separacion vertical
    \item Integracion con sistemas de gestion de inventario y planificacion espacial
\end{itemize}

\subsubsection{Aplicaciones en Inspeccion y Monitoreo}

El sistema 3D es particularmente util para aplicaciones de inspeccion:

\begin{itemize}
    \item Inspeccion de infraestructura vertical (torres, edificios, puentes)
    \item Monitoreo ambiental con mediciones a diferentes altitudes
    \item Mapeo topografico con precision tridimensional
    \item Seguimiento de recursos naturales considerando variaciones temporales y espaciales
\end{itemize}

\section{Conclusiones}

Este proyecto demuestra que la interseccion entre matematicas avanzadas, ingenieria de software y aplicaciones practicas puede generar soluciones innovadoras para desafios tecnologicos contemporaneos. El sistema desarrollado no solo aborda necesidades actuales en el campo de los drones autonomos, sino que tambien establece una base metodologica para futuros desarrollos en sistemas de navegacion autonoma.\\

La implementacion del sistema de visualizacion 3D representa un avance significativo en la comprension espacial de trayectorias elipticas, proporcionando herramientas intuitivas para el analisis de escenarios complejos. La capacidad de visualizar trayectorias en tres dimensiones, junto con la deteccion automatica de intersecciones y conflictos, establece un nuevo estandar para sistemas de gestion de trafico aereo no tripulado.\\

La relevancia del trabajo se extiende mas alla de la aplicacion especifica de drones, proporcionando un framework que puede adaptarse a diversos escenarios donde la coordinacion de multiples agentes moviles es critica. En un mundo cada vez mas automatizado, la capacidad de modelar, predecir y gestionar movimientos complejos en el espacio tridimensional se convierte en una competencia fundamental para la ingenieria del futuro.\\

El enfoque interdisciplinario adoptado, que combina rigor matematico con implementacion practica y consideraciones de usabilidad tanto en 2D como en 3D, ejemplifica las metodologias necesarias para abordar los desafios tecnologicos del siglo XXI. Este proyecto contribuye asi no solo al conocimiento tecnico especifico, sino tambien a la demostracion de buenas practicas en desarrollo de sistemas complejos con interfaces de usuario avanzadas.\\

La integracion exitosa de multiples tecnologias (PyQt6, Tkinter, Matplotlib, NumPy) demuestra la viabilidad de crear sistemas hibridos que aprovechen las fortalezas de diferentes frameworks para lograr resultados superiores. Esta aproximacion metodologica puede servir como modelo para futuros desarrollos en el campo de la visualizacion cientifica y la ingenieria de software.\\

En conclusion, el sistema de modelado de trayectorias elipticas desarrollado, con sus capacidades tanto 2D como 3D, representa un paso significativo hacia la automatizacion segura y eficiente de operaciones de drones, estableciendo las bases para innovaciones futuras que transformaran la manera en que concebimos y gestionamos el transporte aereo autonomo en el espacio tridimensional.

\end{document}